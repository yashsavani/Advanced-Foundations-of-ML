\hypertarget{core-math}{%
\section{Core Math}\label{core-math}}

Before diving into some of the more advanced math concepts, I would
encourage you to at least cursorily go through some of the core math
resources provided here. I recommend doing this even if you think you
are comfortable with all the topics listed below. I can't count the
number of times I thought I completely understood some core math
concept, only to later realize that there was some subtlety I had
neglected. These neglected subtleties can often have a cascading effect
making it very hard to understand some of the more advanced material
that rely on a solid core.

\hypertarget{problem-solving}{%
\subsection{Problem Solving}\label{problem-solving}}

\hypertarget{problems-are-ones-that}{%
\subsubsection{Problems are ones that:}\label{problems-are-ones-that}}

\begin{itemize}
\tightlist
\item
  Engage intellect,
\item
  Make connections to develop a coherent framework,
\item
  Can be solved in more than one way, and
\item
  Should foster effective communication of mathematical ideas.
\end{itemize}

\hypertarget{phases-of-problem-solving}{%
\subsubsection{Phases of problem
solving:}\label{phases-of-problem-solving}}

\begin{itemize}
\tightlist
\item
  \emph{Entry:} what do I know (question, experience), what do I want
  (paraphrase, ambiguities), what can I introduce (diagram, notation).
\item
  \emph{Attack:} brute force, look for patterns.
\item
  \emph{Review:} check, reflect, extend, understand why it works.
\end{itemize}

\hypertarget{propositional-logic}{%
\subsection{Propositional Logic}\label{propositional-logic}}

\begin{itemize}
\tightlist
\item
  Propositional Logic is the foundation for the language of reason.
\item
  A \emph{proposition} is a statement that has one and only one truth
  value, \(\true\) or \(\false\).
\item
  \emph{Propositional variables} are letters used to stand in for actual
  propositional statements. They are usually \(p, q,\) or \(r\).
\item
  \emph{Propositional connectives} are symbols used to connect
  propositional variables together.
\item
  \emph{Propositional expressions} are statements that link together
  propositional variables with potentially multiple propositional
  connectives.
\item
  A \emph{truth table} is a table that exhaustively lists all the
  possible truth values of the expressions.
\item
  \emph{Logical equivalence} (denoted by \(\iff\) or \(\equiv\)) is when
  two propositional expressions are equivalent. That is, they take on
  the same truth value for every possible input.
\end{itemize}

\hypertarget{common-connectives}{%
\subsubsection{Common connectives}\label{common-connectives}}

\begin{itemize}
\tightlist
\item
  \emph{Tautology or \(\top\):} \(\top(p)\) is always \(\true\) no
  matter what truth value \(p\) takes. Usually when we try to prove
  something, we want to show that it is equivalent to a tautology, which
  means it is true no matter what the input truth values.
\item
  \emph{Negation or NOT or \(\lnot\) or \(\bar{p}\):} \(\lnot p\) is
  \(\true\) if and only if \(p\) is \(\false\).
\item
  \emph{Conjunction or AND or \(\land\):} \(p \land q\) is \(\true\) if
  and only if \(p\) and \(q\) are both \(\true\).
\item
  \emph{Inclusive Disjunction or OR or \(\lor\):} \(p \lor q\) is
  \(\true\) if and only if either \(p\) or \(q\) or both are \(\true\).
\item
  \emph{Exclusive Disjunction or XOR or \(\oplus\):} \(p \oplus q\) is
  \(\true\) if \(p\) and \(q\) have different truth values.
\item
  \emph{Implication or Conditional or IMPLIES or \(\to\):} \(p \to q\)
  is \(\false\) if and only if \(p\) is \(\true\) and \(q\) is
  \(\false\).

  \begin{itemize}
  \tightlist
  \item
    \(p\) is called the \emph{antecedent}, and \(q\) is called the
    \emph{consequent}.
  \item
    The \emph{converse} of \(p \to q\) is \(q \to p\). It is not
    equivalent to \(p \to q\).
  \item
    The \emph{inverse} of \(p \to q\) is \(\lnot p \to \lnot q\). It is
    not equivalent to \(p \to q\).
  \item
    The \emph{contrapositive} of \(p \to q\) is \(\lnot q \to \lnot p\).
    It is equivalent to \(p \to q\).
  \end{itemize}
\item
  \emph{Biconditional or IFF or \(\leftrightarrow\):}
  \(p \leftrightarrow q\) is \(\true\) if and only if both \(p\) and
  \(q\) have the same truth value.
\end{itemize}

\hypertarget{truth-table-for-common-connectives}{%
\subsubsection{Truth table for common
connectives}\label{truth-table-for-common-connectives}}

\begin{longtable}[]{@{}cccccccc@{}}
\toprule
\(p\) & \(q\) & \(\lnot p\) & \(p \land q\) & \(p \lor q\) &
\(p \oplus q\) & \(p \to q\) & \(p \leftrightarrow q\) \\
\midrule
\endhead
0 & 0 & 1 & 0 & 0 & 0 & 1 & 1 \\
0 & 1 & 1 & 0 & 1 & 1 & 1 & 0 \\
1 & 0 & 0 & 0 & 1 & 1 & 0 & 0 \\
1 & 1 & 0 & 1 & 1 & 0 & 1 & 1 \\
\bottomrule
\end{longtable}

\hypertarget{order-of-operations}{%
\subsubsection{Order of operations}\label{order-of-operations}}

As a convention, we evaluate any propositional expression in the
following order:

\begin{enumerate}
\def\labelenumi{\arabic{enumi}.}
\tightlist
\item
  Parentheses,
\item
  Negation,
\item
  Conjunction (left to right),
\item
  Inclusive disjunction (left to right),
\item
  Implication, and
\item
  Biconditional.
\end{enumerate}

\hypertarget{de-morgans-theorem}{%
\subsubsection{De Morgan's theorem}\label{de-morgans-theorem}}

One important theorem in propositional logic that appears everywhere is
De Morgan's theorem. The theorem states that:

\begin{align*}
    \lnot(p \land q) &\equiv \lnot p \lor \lnot q, \text{and}\\
    \lnot(p \lor q) &\equiv \lnot p \land \lnot q.
\end{align*}

We can prove this by showing that the truth tables for the expressions
are equivalent.

\hypertarget{set-theory}{%
\subsection{Set Theory}\label{set-theory}}

\begin{itemize}
\tightlist
\item
  A \emph{set} is a collection of members / elements.
\item
  \emph{Roster notation} explicitly lists out every element of the set
  (e.g.~\(A=\{1,2,3,4,5\}, B=\{2,4,6,10\}\)).
\item
  \emph{Set builder notation} implicitly describes all the elements in a
  set
  (e.g.~\(A=\{x \mid x\text{ is an integer between 1 and 5 inclusive}\},\)
  here \(x\) is a dummy variable).
\item
  To denote that an object is an element of a set we use the symbol
  \(\in\) (e.g.~\(3 \in A\)). Note that this is a valid proposition.
\item
  To denote that on object is not an element of a set we use the symbol
  \(\not\in\) (e.g.~\(10 \not\in A\)). Note that this is also a valid
  proposition.
\item
  Elements cannot appear more than once in a set, though they can appear
  more than once in a multiset.
\item
  A set is not ordered.
\item
  The \emph{universal set} is the set of all the objects being
  considered. It is usually notated as \(\mathcal{U}\). In our case we
  may assume that \(\mathcal{U} = \{x \mid x\) is an integer between 1
  and 10 inclusive \(\}\)
\item
  The \emph{null set or empty set} is the set containing no elements. It
  is notated as \(\varnothing = \{ \}\).
\item
  Set theory has operators to combine sets.
\item
  \emph{Venn diagrams} are a visual representation of sets and the
  operations on sets.
\end{itemize}

\hypertarget{set-operators}{%
\subsubsection{Set operators}\label{set-operators}}

\begin{itemize}
\item
  \emph{Complement or \(^c\):} \(A^c = \{ x \mid \lnot(x \in A) \}\)
  (e.g.~\(A^c = \{6, 7, 8, 9, 10\}\)).
\item
  \emph{Intersection or \(\cap\):}
  \(A \cap B = \{x \mid x \in A \land x \in B\}\)
  (e.g.~\(A \cap B = \{2,4\}\)). Note that the intersection is
  \emph{associative}. That is,
  \((A \cap B) \cap C = A \cap (B \cap C) = A \cap B \cap C\).
\item
  \emph{Union or \(\cup\):}
  \(A \cup B = \{x \mid x \in A \lor x \in B\}\)
  (e.g.~\(A \cup B = \{1,2,3,4,5,6,8,10\}\)). Note that the union is
  \emph{associative}. That is,
  \((A \cup B) \cup C = A \cup (B \cup C) = A \cup B \cup C\).
\item
  \emph{Difference or \(\setminus\):}
  \(A \setminus B = \{x \mid x \in A \land x \not\in B\}\)
  (e.g.~\(A \setminus B = \{1,3,5\}\)).
\item
  \emph{Symmetric difference or \(\Delta\):}
  \(A \Delta B = \{x \mid x \in A \oplus x \in B\}\)
  (e.g.~\(A \Delta B = \{1,3,5,6,8,10\}\)).
\end{itemize}

\hypertarget{subsets-supersets-and-equality}{%
\subsubsection{Subsets, supersets, and
equality}\label{subsets-supersets-and-equality}}

Three other propositional statements for sets are the subset, superset,
and equality relations.

\begin{itemize}
\tightlist
\item
  We say that a set \(D\) is a subset \emph{subset} of a set \(A\), if
  every element \(x \in D\) is also an element of \(A\). Formally, we
  can write this as \(\forall x, x \in D \to x \in A\). It is notated as
  \(D \subseteq A\).
\item
  We say that a set \(A\) is a \emph{superset} of a set \(D\), if every
  element \(x \in D\) is also an element of \(A\). the superset
  proposition is essentially the reverse of the subset proposition. We
  notate this by \(A \supseteq D\).
\item
  We say that a set \(A\) is \emph{equivalent} to a set \(D\) if it is
  both a superset and a subset of \(D\). Formally, we can write this as
  \(\forall x, x\in D \leftrightarrow x \in A\). We notate this by
  \(A = D\).
\end{itemize}

We say that \(D\) is a strict subset of \(A\) if \(D \subseteq A\), but
\(D \not= A\). In this case we write \(D \subset A\). We can reverse the
statement to get strict supersets (\(A\supset D\)).

A trivial statement that is always true is that \(\varnothing \in C\)
for all possible sets \(C\). We say that this is true vacuously or true
in an empty sort of way.

Another such trivial statement is that every set is a subset of itself
since it is equal to itself.

\hypertarget{special-sets}{%
\subsubsection{Special sets}\label{special-sets}}

Some special sets that we often consider are:

\begin{itemize}
\tightlist
\item
  The set of \emph{natural numbers}, \(\N = \{1,2,3,4,\ldots\}\) (some
  people claim \(0 \in \N\)).
\item
  The set of \emph{integers}, \(\Z = \{\ldots,-2,-1,0,1,2,\ldots\}\).
\item
  The set of \emph{rational numbers},
  \(\Q = \left\{x \mid x = \frac{p}{q}, \where p,q \in \Z \right\}\).
\item
  The set of \emph{real numbers}, \(\R\) that are the completion of
  \(\Q\).
\item
  The set of \emph{irrational numbers} given by \(\R \setminus \Q\).
\item
  The set of \emph{complex numbers},
  \(\C = \{x \mid x = a + ib, \where a,b \in \R \land i = \sqrt{-1}\}\).
\end{itemize}

From the definitions we can see that
\(\N \subset \Z \subset \Q \subset \R \subset \C\).

\hypertarget{russells-paradox-families-of-sets-and-index-sets}{%
\subsubsection{Russell's paradox, families of sets, and index
sets}\label{russells-paradox-families-of-sets-and-index-sets}}

A set cannot contain itself. If we let this be the case then we get a
paradox. The paradox comes from the set \(A = \{X \mid X \not\in X \}\).
That is, the set that contains all sets which do not contain themselves.
If \(A\) is an element of \(A\), then \(A\) must not be in the set, but
then it must be in the set. This leads to a contradiction. To avoid this
kind of paradox, we accept a system of 9 axioms governing everything we
can call a set and everything we can do with a set. We call this set of
axioms ZFC or Zermelo--Fraenkel set theory with the axiom of choice.

We call a set comprised of other sets a family of sets. We usually
denote such families of sets with a uppercase script character like
\(\mathcal{F}\).

Index sets are sets that are used to enumerate the elements of a set,
though they don't need to be a set of numbers. They are sometimes used
with families of sets to enumerate the sets contained within the family.
For example,
\(\mathcal{F} = \{L_1, L_2, \ldots\}, \where L_i = \{p \mid p \in \mathcal{P} \land p \leq i\}\),
so \(L_{10} = \{2,3,5,7\}\). An index set might be
\(\mathcal{I} = \{1,2,\ldots,100\}\) and it can be used to index the
\(L_i\) sets of \(\mathcal{F}\) for \(i \in \mathcal{I}\).

\hypertarget{predicate-logic}{%
\subsection{Predicate Logic}\label{predicate-logic}}

\begin{itemize}
\tightlist
\item
  Linguistically a proposition needs to be a declarative sentence with a
  subject, which is a noun phrase, and a predicate, which is a verb
  phrase.
\item
  We can represent the subject by an element of a set and store the
  predicate separately (e.g.~\(p(x) =\) ``x is an even number'', where
  \(x\) is a free variable and \(p\) is the predicate).
\item
  The \emph{truth set} of \(p\) is the set of all \(x\) for which
  \(p(x)\) is \(\true\). That is, the truth set of \(p\) is
  \(\{x \mid p(x)\}\).
\item
  \emph{Quantifiers} tell us how big our truth set is.
\item
  The \emph{universal quantifier or ``for all x, p(x)'' or
  \(\forall x, p(x)\)} says that \(p(x)\) is \(\true\) for every
  possible element of our universal set.

  \begin{itemize}
  \tightlist
  \item
    \((\forall b \in B, p(b)) \iff (\forall b, (b \in B \to p(b)))\) is
    a useful identity for when we want to say something about every
    element in a non universal set \(B\).
  \end{itemize}
\item
  The \emph{existential quantifier or ``there exists x such that p(x)''
  or \(\exists x, \st p(x)\)} says that there exists an \(x\) in the
  universal set such that p(x) is \(\true\).

  \begin{itemize}
  \tightlist
  \item
    \((\exists a \in A, \st p(a)) \iff (\exists a, \st (a \in A \land p(a)))\)
    is a useful identity for when we want to say something about some
    element of a non universal set \(A\).
  \item
    Sometimes we want to say that there exists a unique element for
    which \(p(x)\) is \(\true\). In this case we use the notation
    \(\exists! x, \st p(x)\) to say that there exists exactly one
    element \(x \in \mathcal{U}\) that is in the truth set of \(p(x)\).
  \end{itemize}
\item
  We can see that
  \(\forall b \in B, p(b) \implies \exists b \in B, \st p(b)\) is
  trivially true. This tells us that any statement with a universal
  quantifier is a stronger than one with an existential quantifier.
\item
  We can also combine quantifiers to get compound quantifiers.

  \begin{itemize}
  \tightlist
  \item
    \(\forall x, \exists y, \st (x < y)\) is \(\true\) for all natural
    numbers.
  \item
    \(\exists y, \st \forall x, \st (x < y)\) is \(\false\) for all
    natural numbers since there is no largest natural number.
  \end{itemize}
\item
  We can also negate quantifiers.

  \begin{itemize}
  \tightlist
  \item
    \(\lnot \forall x, p(x) \iff \exists x, \st \lnot p(x)\).
  \item
    \(\lnot \exists x, \st p(x) \iff \forall x, \lnot p(x)\).
  \end{itemize}
\item
  For a family of sets \(\mathcal{F}\), we say that
  \(\bigcup_{A \in \mathcal{F}} A = \{a | \exists A \in \mathcal{F}, \st a \in A\}\)
\item
  For a family of sets \(\mathcal{F}\), we say that
  \(\bigcap_{A \in \mathcal{F}} A = \{a | \forall A \in \mathcal{F}, a \in A\}\)
\end{itemize}

\hypertarget{proof-techniques}{%
\subsection{Proof Techniques}\label{proof-techniques}}

\begin{itemize}
\tightlist
\item
  \textbf{Proofs} are demonstrations that if some fundamental statements
  are assumed to be \(\true\), then some consequent mathematical
  statements must necessarily also be \(\true\).
\item
  \textbf{Definitions} are accurate, reversible descriptions of
  concepts.
\item
  \textbf{Axioms or postulates} are statements that we accept without
  proof. All proofs must only be built on these axioms.
\item
  \textbf{Theorems} are statements that we prove using our axioms and
  other previously proved theorems (e.g.~Pythagorean theorem).
\item
  \textbf{Lemmas} are mini-theorems that we use to prove the main
  theorems. Though, it is not necessary that the lemma be simple to
  prove.
\item
  \textbf{Corollaries} are results that follow easily from a theorem.
\item
  \textbf{Conjectures} are ideas that are believed to be \(\true\) but
  are not yet proven (e.g.~Goldbach's conjecture, Collatz conjecture,
  Fermat's conjecture).
\item
  \textbf{Trivial} solutions are obvious solutions. Though, sometimes
  mathematicians claim triviality even when the solutions are not so
  obvious, especially for beginners.
\item
  A fundamental mathematical question worth asking is if there is a set
  of foundational mathematical axioms upon which all of mathematics can
  be built. Towards this goal, we desire a mathematical system that has
  the following properties:

  \begin{itemize}
  \tightlist
  \item
    \textbf{Completeness} is when all true mathematical statements can
    be proven using the axioms.
  \item
    \textbf{Consistency} is when the axioms do not lead to any
    contradictions.
  \item
    \textbf{Decidability} is when every statement can be identified as a
    valid mathematical statement.
  \end{itemize}
\item
  Completeness, consistency, and decidability (Godel's incompleteness
  theorems),
\item
  Valid arguments (modus ponens, modus tollens, law of syllogism),
\item
  Invalid arguments (affirming the consequent),
\item
  Direct proof,
\item
  Proof by contrapositive,
\item
  Proof by contradiction,
\item
  Proof of conjunction,
\item
  Proof of inclusive disjunction,
\item
  Universal: let x be arbitrary, prove p(x),
\item
  Existential: find an x such that p(x) is true,
\item
  Uniqueness: assume p(x), p(y), show x = y,
\item
  Successor operation,
\item
  Proof by induction (weak and strong), and
\item
  Deductive vs inductive reasoning.
\end{itemize}

\hypertarget{relations}{%
\subsection{Relations}\label{relations}}

\begin{itemize}
\tightlist
\item
  Tuples,
\item
  Cartesian products (cartesian square, cartesian power),
\item
  Arity of operations,
\item
  Domain and codomain,
\item
  Inverse relations,
\item
  Graphs and digraphs,
\item
  Relations on a set,
\item
  Pre-order relations (reflexive, transitive),
\item
  Equivalence relations (reflexive, transitive, symmetric),
\item
  Equivalence classes,
\item
  Partial-order relations (reflexive, transitive, antisymmetric)
  (posets),
\item
  Hasse diagrams,
\item
  Total-order relations (complete, transitive, antisymmetric),
\item
  Strict-order relations (reflexive -\textgreater{} irreflexive)
  (trichotomous),
\item
  Upper bounds, lower bounds, maximum, minimum, and
\item
  Least upper bound (supremum), greatest lower bound (infimum).
\end{itemize}

\hypertarget{functions}{%
\subsection{Functions}\label{functions}}

\begin{itemize}
\tightlist
\item
  Left-total, left-unique, right-total, right-unique,
\item
  Function (left-total and right-unique),
\item
  Prototype and definition,
\item
  Images, ranges, and preimages,
\item
  Injective/one-to-one (left-unique),
\item
  Surjective/onto (right-total),
\item
  Bijective/one-to-one correspondence (right-total and left-unique),
\item
  Inverse functions,
\item
  Function composition,
\item
  Restricting domain and codomain, and
\item
  Monotonicity.
\end{itemize}

\hypertarget{number-systems}{%
\subsection{Number Systems}\label{number-systems}}

\begin{itemize}
\tightlist
\item
  Cardinality,
\item
  Pigeon-hole principle,
\item
  Cantor-Bernstein-Schroeder theorem,
\item
  Finite Sets,
\item
  Peano's axioms,
\item
  Countable sets (Aleph-Null),
\item
  Integers,
\item
  Rationals (snaking argument),
\item
  Arithmetic operations,

  \begin{itemize}
  \tightlist
  \item
    Addition with inverse (subtraction),
  \item
    Multiplication (repeated addition) with inverse (division),
  \end{itemize}
\item
  Exponentiation (repeated multiplication) with inverses (logarithms and
  radicals/roots since exponentiation is not commutative),
\item
  Algebraic numbers,
\item
  Real Numbers

  \begin{itemize}
  \tightlist
  \item
    Zenos' paradoxes,
  \item
    Transcendental numbers,
  \item
    Dedekind cut,
  \item
    Supremum property, and
  \item
    Cantor's diagonalization argument,
  \end{itemize}
\item
  Cardinality of the continuum (continuum hypothesis),
\item
  Hilbert's paradox of the Grand Hotel,
\item
  Complex numbers (closure under roots) (no ordering),

  \begin{itemize}
  \tightlist
  \item
    Complex arithmetic,
  \item
    Conjugate,
  \item
    Modulus,
  \item
    Argand diagram, and
  \end{itemize}
\item
  Quarternions (not commutative) and octonions (not associative).
\end{itemize}

\hypertarget{basic-number-theory}{%
\subsection{Basic Number Theory}\label{basic-number-theory}}

\begin{itemize}
\tightlist
\item
  Divisibility,
\item
  Multiples and factors,
\item
  Euclidean division (dividend = quotient divisor + remainder),
\item
  Greatest common divisor, and lowest common multiple,
\item
  Relative primes,
\item
  Euclid's algorithm,
\item
  Linear combination,
\item
  Diophantine equations,
\item
  Prime numbers,
\item
  Unique prime factorization,
\item
  Euler totient function,
\item
  Prime counting function,
\item
  Sieve of Eratosthenes,
\item
  Mersenne primes,
\item
  Mill's theorem, and
\item
  Riemann Hypothesis.
\end{itemize}

\hypertarget{modular-arithmetic}{%
\subsection{Modular Arithmetic}\label{modular-arithmetic}}

\begin{itemize}
\tightlist
\item
  Remainders follow cyclic pattern,
\item
  Congruence modulo m,
\item
  Properties of congruence,
\item
  Congruence classes,
\item
  Canceling when dividing by relative prime of modulus,
\item
  Divisibility tests, and
\item
  Multiplicative inverses.
\end{itemize}

\hypertarget{birkhoff-geometry-geometry-based-on-real-numbers}{%
\subsection{Birkhoff Geometry (geometry based on real
numbers)}\label{birkhoff-geometry-geometry-based-on-real-numbers}}

\begin{itemize}
\tightlist
\item
  Undefined terms

  \begin{itemize}
  \tightlist
  \item
    point,
  \item
    line (set of points),
  \item
    distance (positive definite, symmetric, triangle inequality), and
  \item
    angle (formed by 3 points).
  \end{itemize}
\item
  Definitions

  \begin{itemize}
  \tightlist
  \item
    Between,
  \item
    Line segment,
  \item
    Half-line/ray and endpoint,
  \item
    Parallel,
  \item
    Straight angle, right-angle, perpendicular,
  \item
    Triangle, vertices, degenerate triangle, and
  \item
    Similar and congruent.
  \end{itemize}
\item
  Postulates

  \begin{itemize}
  \tightlist
  \item
    Bijection between points of a line and real numbers,
  \item
    Unique line containing two distinct points,
  \item
    Bijection between rays and real numbers mod 2 pi, and
  \item
    Triangle congruences.
  \end{itemize}
\item
  Theorems

  \begin{itemize}
  \tightlist
  \item
    Linear pair,
  \item
    Vertical angles,
  \item
    Triangle angles sum,
  \item
    Corresponding angles (and converse),
  \item
    Alternate interior/exterior angles (and converses),
  \item
    Interiors on the same side (and converse),
  \item
    Perimeter and Area,
  \item
    Parallelogram,
  \item
    Rectangle,
  \item
    Rhombus,
  \item
    Square,
  \item
    Trapezoid,
  \item
    Polygons (regular and irregular),
  \item
    Circles (radius and diameter, chords, tangents, arcs, angles),
  \end{itemize}
\item
  3D cylinders and volume,
\item
  Planes and disks,
\item
  Spheres,
\item
  Polyhedrons are 3D object with flat faces and edges, and
\item
  Dihedrons are a 2D object in 3D space with two faces and edges of
  negligible thickness.
\item
  Trigonometry

  \begin{itemize}
  \tightlist
  \item
    Right-angle triangle (opposite, adjacent, hypotenuse),
  \item
    Sine, cosine, tangent,
  \item
    Reciprocals (cosecant, secant, cotangent),
  \item
    Inverses (arcsine, arccosine, arctangent)
  \item
    Tan = sin / cos,
  \item
    Unit circle,
  \item
    Radians vs degrees,
  \item
    Domains and ranges,
  \item
    Periods,
  \item
    Pythagorean identities (\(\sin^2 + \cos^2 = 1\))
  \item
    Sine is odd, cos is even,
  \item
    Sum and difference formulae (cos together and flip, sin separate and
    keep),
  \item
    Product to sum and sum to product formulae,
  \item
    Convert sine to cosine by phase shift,
  \item
    Relation to complex numbers
    (\(e^{ix} = \cos(x) + i \cdot \sin(x)\)),
  \item
    Add complex numbers using parallelogram rule, and
  \item
    Multiply complex numbers by adding the angles and multiplying the
    lengths.
  \end{itemize}
\end{itemize}

\hypertarget{systems-of-equations-matrices-polynomials}{%
\subsection{Systems of Equations, Matrices,
Polynomials}\label{systems-of-equations-matrices-polynomials}}

\begin{itemize}
\tightlist
\item
  Algebraic expressions (rational exponents),
\item
  Linear equations and inequalities,
\item
  Linear graphs,
\item
  Critical points,
\item
  Multiple equations with multiple unknowns,
\item
  Writing linear systems of equations as matrices,
\item
  Matrix multiplication,
\item
  Matrix inverses (singular matrices),
\item
  Absolute value and piecewise equations,
\item
  Monomials,
\item
  Polynomials,
\item
  Quadratic equation,
\item
  Cubic equation,
\item
  No closed form for quartic equation (Galois),
\item
  Binomials,
\item
  Binomial theorem (permutations and combinations) (Pascal's triangle),

  \begin{itemize}
  \tightlist
  \item
    \((a-b)^2 = (a-b)(a+b)\),
  \end{itemize}
\item
  Foil,
\item
  Polynomial arithmetic,
\item
  Polynomial factorization,
\item
  Polynomial division,
\item
  Partial fractions (Partial fraction decomposition),
\item
  Polynomial graphs,
\item
  Reciprocal graph,
\item
  Complex polynomials, and
\item
  Fundamental theorem of algebra (any polynomial of degree d has d,
  possibly complex, roots) (connects algebra with geometry).
\end{itemize}

\hypertarget{group-theory}{%
\subsection{Group Theory}\label{group-theory}}

\begin{itemize}
\tightlist
\item
  Group (closed, associative, unique identity, unique inverse),
\item
  Idempotence,
\item
  Order of a group is the cardinality of the group,
\item
  Finite and infinite groups,
\item
  GLnR group of nonsingular matrices under matrix multiplication,
\item
  Abelian groups (commutative group),
\item
  Cayley table,
\item
  Subgroup (show closure and inverse for infinite group, only closure
  for finite group),
\item
  Exponentiation,
\item
  Generated group,
\item
  Order of an element,
\item
  Cyclic group,
\item
  Dihedral group (symmetries of a regular polygon), and
\item
  Symmetric/permutation group (bijections from a set to itself)
  (transpositions, cycles, composition).
\end{itemize}

\hypertarget{rings-and-fields}{%
\subsection{Rings and Fields}\label{rings-and-fields}}

\begin{itemize}
\tightlist
\item
  Rings are additive Abelian groups that are closed under
  multiplication,

  \begin{itemize}
  \tightlist
  \item
    Addition and multiplication are linked through the distributive
    property,
  \end{itemize}
\item
  Exponentiation is repeated multiplication,
\item
  0a = a0 = 0,
\item
  a (-b) = (-a) b = -(ab),
\item
  -(-a)(-b) = ab,
\item
  Infinite and finite rings,
\item
  Polynomial rings,
\item
  Commutative rings,
\item
  Unital ring (multiplicative identity) (inverses and units),

  \begin{itemize}
  \tightlist
  \item
    1 = 0 implies the ring is the trivial ring,
  \end{itemize}
\item
  Division ring (unital ring where every non-zero element is a unit),
\item
  Zero divisor (non-zero element times another non-zero element gives
  0),
\item
  Intersection of units and zero divisors is empty,
\item
  Integral domain (commutative, unital ring with no zero divisors),
\item
  Can cancel factors from both sides of an integral domain,
\item
  A field is an algebraic structure that is Abelian under both addition
  and multiplication,

  \begin{itemize}
  \tightlist
  \item
    A field is an integral domain where every non-zero element is a
    unit,
  \end{itemize}
\item
  Rationals, reals, and complex numbers are infinite fields,
\item
  Integers modulo a prime are finite fields, and
\item
  All arithmetic operations \(+, -, \times\), and \(\div\) play nicely
  together.
\end{itemize}

\hypertarget{morphisms}{%
\subsection{Morphisms}\label{morphisms}}

\begin{itemize}
\tightlist
\item
  Symmetry is patterned self-similarity,
\item
  Dihedral group \(D_n\) is a non-abelian, non-cyclic group with order
  \(2n\),
\item
  Homomorphisms (operation preserving functions),
\item
  Isomorphisms (bijective homomorphisms) (equivalence relation),
\item
  Endomorphisms and automorphisms (isomorphic endomorphism),
\item
  Automorphism group,
\item
  Group action,
\item
  Any finite cyclic group of order m is isomorphic to the integers mod
  m.
\item
  Cayley's theorem (every group is isomorphic to a subgroup of the
  symmetric group).
\item
  Every permutation can be expressed as a composition of transpositions,
  and
\item
  Every permutation has a fixed parity (odd or even based on number of
  transpositions).
\end{itemize}

\hypertarget{single-variable-calculus}{%
\subsection{Single Variable Calculus}\label{single-variable-calculus}}

\begin{itemize}
\tightlist
\item
  Zenos' paradoxes,
\item
  Sequences (notation, inductive and recursive definition),
\item
  Limits (intuitive definition),

  \begin{itemize}
  \tightlist
  \item
    Limits at infinity and infinite limits,
  \item
    One sided limits,
  \end{itemize}
\item
  Convergence and divergence (periodic),
\item
  Squeeze theorem,
\item
  Properties of limits,

  \begin{itemize}
  \tightlist
  \item
    Constant scaling,
  \item
    Closure under arithmetic operations (\(+, -, \times, \div\), powers,
    and roots),
  \end{itemize}
\item
  Limit evaluation,
\item
  Continuous functions and closure of continuity under composition,
\item
  Intermediate value theorem,
\item
  Derivative,

  \begin{itemize}
  \tightlist
  \item
    Interpretation as relative changes in variables,
  \end{itemize}
\item
  L'Hopital's rule,
\item
  Polynomial derivative,
\item
  Basic properties of derivatives,

  \begin{itemize}
  \tightlist
  \item
    Constant derivative is zero,
  \item
    Constant scaling,
  \item
    Addition, subtraction,
  \item
    Product rule,
  \item
    Quotient rule,
  \item
    Power rule,
  \item
    Chain rule,
  \end{itemize}
\item
  Exponential derivative and Euler's number,
\item
  Higher order derivatives (smooth functions),
\item
  Implicit differentiation (logarithmic derivative),
\item
  Common derivatives,

  \begin{itemize}
  \tightlist
  \item
    c, x, sin, cos, tan, arcsin, arccos, arctan, e\^{}x, a\^{}x, ln(x),
  \end{itemize}
\item
  Optimization (constrained and unconstrained),
\item
  Critical points and extrema (local and global),
\item
  Convex and concave functions,
\item
  Extreme value theorem,
\item
  1st and 2nd derivative test,
\item
  Lagrange multipliers,
\item
  Newton's method,
\item
  Mean value theorem,
\item
  Riemann integral and anti-derivative,
\item
  Fundamental theorem of calculus,
\item
  Definite vs indefinite integral,
\item
  Properties of anti-derivative,

  \begin{itemize}
  \tightlist
  \item
    Constant scaling,
  \item
    Addition and subtraction,
  \end{itemize}
\item
  Common integrals,

  \begin{itemize}
  \tightlist
  \item
    k, polynomials, 1/x, ln(x), e\^{}x, sin, cos, tan
  \end{itemize}
\item
  Standard integral techniques,

  \begin{itemize}
  \tightlist
  \item
    Substitution,
  \item
    Integration by parts,
  \end{itemize}
\item
  Partial sums and infinite series,

  \begin{itemize}
  \tightlist
  \item
    Absolutely convergence,
  \item
    Conditionally convergence,
  \end{itemize}
\item
  Arithmetic series (average of the terms times the number of terms),
\item
  Geometric series (infinite limit = a/(1-r)),
\item
  Telescoping series,
\item
  Harmonic series (infinitely divergent) and p-series,
\item
  Convergence tests,

  \begin{itemize}
  \tightlist
  \item
    Integral test,
  \item
    (Limit) comparison test,
  \item
    Alternating series test,
  \item
    Ratio test,
  \item
    Root test,
  \item
    Power series (radius of convergence), and
  \end{itemize}
\item
  Taylor series,

  \begin{itemize}
  \tightlist
  \item
    1/(1-x) = sum x\^{}i,
  \item
    e\^{}x = sum x\^{}i / i!,
  \item
    cos x = sum (-1)\^{}n x\^{}\{2n\}/(2n)!,
  \item
    sin x = sum (-1)\^{}n x\^{}\{2n+1\}/(2n+1)!.
  \end{itemize}
\end{itemize}

One excellent resource I have found for some of this material is the
playlist
\href{https://youtube.com/playlist?list=PLZzHxk_TPOStgPtqRZ6KzmkUQBQ8TSWVX}{Introduction
to Higher Mathematics}. I recommend going through this playlist at least
once.

A crucial tool in the machine learning toolbox is calculus. Most of the
machine learning literature makes extensive use of multivariable
calculus. However, before diving into the nuances of multivariable
calculus, it's important to make sure you have a solid foundation and
intuition for single variable calculus first. To gain the intuition I
would recommend a
\href{https://www.youtube.com/playlist?list=PLZHQObOWTQDMsr9K-rj53DwVRMYO3t5Yr}{this
playlist} by 3Blue1Brown, or
\href{https://www.khanacademy.org/math/ap-calculus-bc}{this course} on
AP calculus from Khan Academy. If you already have a strong calculus
foundation, you can safely skip this section. Though, I would still
encourage you to go through
\href{https://www.youtube.com/playlist?list=PLZHQObOWTQDMsr9K-rj53DwVRMYO3t5Yr}{the
essence of calculus playlist} by 3Blue1Brown at some point though. It
helps fill in a lot of intuition that more formal classes may have
missed. Who knows? You may learn something you didn't realize you had
missed. Even small insights in the foundations can lead to major
revelations later on.
