\hypertarget{introduction}{%
\section{Introduction}\label{introduction}}

There are already several repositories and courses that cover the
foundations of Machine Learning. Here are a few of them:

\begin{itemize}
\tightlist
\item
  https://github.com/jonkrohn/ML-foundations, and
\item
  https://github.com/dontless/Machine-Learning-Foundations-A-Case-Study-Approach.
\end{itemize}

Most of these resources are designed to get you to the starting line.
They provide just enough material for you to get a job doing machine
learning, or for you to get started on machine learning projects.
However, if you have ever tried reading a theoretical machine learning
paper from a conference like
\href{http://www.learningtheory.org/colt2021/accepted.html}{COLT} or one
of the more advanced applied papers from a conference like
\href{https://icml.cc/Conferences/2021/Schedule?type=Poster}{ICML}, it
is usually clear that a lot is missing from these ``foundational''
curricula.

Some might argue that the best way to learn this missing material is to
just dive in and start reading papers. Try to figure out the missing
concepts on the fly. That is exactly what I have been doing over the
past few years, and I am sad to say that this approach is very
insufficient. While it may get you to a place where you can understand
the research superficially, this top-down model leaves you with a deep
sense of inadequacy; a feeling that you are lacking something base. With
this approach, you eventually realize that you lack a cohesive narrative
and many foundational concepts. You find it difficult to see the work in
a wider theoretical and applied context, and have a hard time
identifying nontrivial extensions. The need to go back and re-learn some
of the mathematical foundations then becomes imperative. This is exactly
what I went through, and if you have had a similar experience, I hope
this resource will be of some help to you.

If you have no experience with machine learning at all, then I would
encourage you to take an intro class in machine learning to see just how
cool this field is. One example of such a course is
(https://www.coursera.org/learn/machine-learning) by Andrew Ng. Machine
learning is one of the most exciting fields to have blossomed over the
past decade, and there are still a plethora of untapped applications to
many of the existing techniques. I urge you to explore as much as you
can, and also to start reading papers and thinking of extensions to
contemporary work. This material should be seen more as a complement
than a prerequisite to your machine learning journey. If you find
yourself struggling with some of the advanced material, or you get a
gnawing sense that you are missing something fundamental while reading
the literature, then maybe something from this resource can help you.

For those of you who already have some experience with machine learning,
I hope this resource can act as a good reference and review checklist.
There are many topics and perspectives to consider any machine learning
problem through. While you may be familiar with some of the ideas listed
here, I hope this repository presents a few alternative viewpoints that
you may have neglected to consider your problems through. Hopefully,
these alternative lenses can highlight some key, deep insights into your
problem that would otherwise have lied dormant. Furthermore, having a
systematic checklist for review items can also help focus and prioritize
future explorations. By going through the topics listed here, you can
identify areas of high impact that you may not be completely comfortable
with. Focusing on these areas has the potential to maximize the utility
of your explorations.

My goal with this repository is to create a curated list of resources
for those who want to do a deeper dive into some of the more advanced
foundations of machine learning. I hope that anyone with a high school
background in math and CS who goes through all the material here will
successfully be able to read and carry out state-of-the-art research in
both theoretical and applied machine learning feeling empowered and
confident in their work.

The rest of this document is organized into several different
sub-fields, with accompanying learning resources, that I think are
necessary to gain this advanced foundation in machine learning.
