\hypertarget{core-cs}{%
\section{Core CS}\label{core-cs}}

Many of the recent innovations in machine learning can be traced
directly back to computer science roots. Furthermore, much of the
language used in the contemporary machine learning literature is
inherited almost entirely from the computer science literature. Also,
ultimately most machine learning ideas will have to be transcribed into
code that will run on computers. Therefore, having a solid foundation in
computer science will be imperitave to your success in machine learning
research. All the topics covered here should be part of any introductory
CS curriculum.

\hypertarget{programming-basics}{%
\subsection{Programming basics}\label{programming-basics}}

\begin{itemize}
\tightlist
\item
  Types,
\item
  Variables,
\item
  Arrays and strings,
\item
  Conditionals,
\item
  Loops,
\item
  Functions,
\item
  Bits and bitwise operations,
\item
  Recursion and backtracking,
\item
  Classes and objects,
\item
  Functional vs object oriented paradigms, and
\item
  Pointers/references and memory organization (stack and heap).
\end{itemize}

\hypertarget{algorithms}{%
\subsection{Algorithms}\label{algorithms}}

\begin{itemize}
\tightlist
\item
  Search,
\item
  Sort,
\item
  Divide and Conquer,
\item
  Bachmann-Landau/asymptotic notation (Big/Little O, Omega, Theta),
\item
  Master theorem,
\item
  Randomized algorithms,
\item
  Graphs/networks,
\item
  Minimum spanning trees,
\item
  Balancing trees,
\item
  Depth first search and breadth first search,
\item
  Dijkstra,
\item
  Dynamic programming (Bellman Ford, Floyd Warshall, Knapsack),
\item
  Min-cut and max-flow (Ford-Fulkerson), and
\item
  P vs NP (NP-completeness, 3 SAT, reductions).
\end{itemize}

\hypertarget{data-structures}{%
\subsection{Data Structures}\label{data-structures}}

\begin{itemize}
\tightlist
\item
  Lists,
\item
  Stacks,
\item
  Queues (priority queues),
\item
  Trees,
\item
  Heaps,
\item
  Hashmaps (amortization), and
\item
  Sets.
\end{itemize}

\hypertarget{systems-and-additional-topics}{%
\subsection{Systems and Additional
Topics}\label{systems-and-additional-topics}}

\begin{itemize}
\tightlist
\item
  Hardware overview (transistors, logic gates, latches, memory, CPU,
  RAM, GPU, peripherals),
\item
  Unix (filesystem, interrupts/signals, system calls, processess,
  interprocess communication, terminal, shell scripting),
\item
  Multiprocessing vs multithreading, and asynchronous processing
  (mutexes, semaphores, threadpools, race conditions, deadlocks),
\item
  Networking (TCP/IP, sockets, HTTP, HTML5, CSS3, bandwidth, latency,
  throughput),
\item
  Standard libraries (string manipulation, file I/O, datetime, basic
  arithmetic and math operations),
\item
  Regular expressions,
\item
  Debugging techniques, and
\item
  Profiling.
\end{itemize}

I learned most of this material through various online and in person
classes over 13 years ago. As a result, most of the sources I used have
since become antiquated and have been replaced with several better and
friendlier versions. I am happy to add more resources to this section
based on recommendations.

For anyone who has never programmed before, I recommend going through
\href{https://www.youtube.com/watch?v=KkMDCCdjyW8\&list=PL84A56BC7F4A1F852}{Programming
Methodology} by Prof.~Mehran Sahami. While this playlist is on the older
side, I think Prof.~Sahami does a fantastic job of laying out the basics
of programming in this course. This can safely be skipped if you already
have some experience with programming.

Once you have whet your appetite for programming and basic computer
science,
\href{https://www.youtube.com/watch?v=FIroM06V2MA\&list=PL-h0BZdG_K4kAmsfvAik-Za826pNbQd0d}{CS106b}
is a good segue into the world of algorithms and abstractions. If you
are already familiar with most of the basic algorithms and abstractions
listed above feel free to skip this course as well. If you think your
algorithms and abstraction skills may have gotten a little rusty take a
look at the
\href{https://web.stanford.edu/class/archive/cs/cs106b/cs106b.1214/}{slides}
from the version of the course taught by Keith Schwarz.

Now that you have had some experience with programming and learning how
to abstract some of the basic ideas from code into general algorithms,
it's important to learn some of the most fundamental algorithms and
algorthimic analysis techniques in computer science. To cover this
material there are several fantastic resources. -
\href{https://www.youtube.com/watch?v=hbJMUzZtJgk\&list=PLyhSTP3Z5_mZ8krUa2JsvL7V755ogHgkK}{CS161}
by Prof.~Tim Roughgarden is a great introduction to many of the most
important algorithmic techniques you will need to know. - An equivalent
alternative to the CS161 lectures, is a multi-part course by
Prof.~Roughgarden on Coursera split into
\href{https://www.coursera.org/learn/algorithms-divide-conquer}{part 1
(divide and conquer)},
\href{https://www.coursera.org/learn/algorithms-graphs-data-structures}{part
2 (graph search, shortest paths)},
\href{https://www.coursera.org/learn/algorithms-greedy}{part 3 (greedy
algorithms)}, and
\href{https://www.coursera.org/learn/algorithms-npcomplete}{part 4
(NP-completeness)} also taught by Prof.~Roughgarden. If you prefer,
\href{https://www.youtube.com/playlist?list=PLXFMmlk03Dt7Q0xr1PIAriY5623cKiH7V}{playlist
1} and
\href{https://www.youtube.com/playlist?list=PLXFMmlk03Dt5EMI2s2WQBsLsZl7A5HEK6}{playlist
2} are the corresponding YouTube playlists. I recommend this option over
the CS161 lectures because of the audio quality. - A course series on
introductory algorithms from MIT that covers a little more than CS161
(\href{https://ocw.mit.edu/courses/electrical-engineering-and-computer-science/6-006-introduction-to-algorithms-spring-2020/index.htm}{6.006}
and
\href{https://ocw.mit.edu/courses/electrical-engineering-and-computer-science/6-046j-design-and-analysis-of-algorithms-spring-2015/index.htm}{6.046}).
I preferred the dynamic programming lectures in this series.

Regardless of whether you choose to pursue theoretical or applied
research, you will inevitably come across some literature that requires
you to have a working systems background. The modern revolution in
machine learning was at least in some part a consequence of the
incredible advances we have made in hardware technology. Understanding
the relationship between hardware and algorithms therefore becomes
critical to an appreciation of modern machine learning.

For this basic foundation in systems, I would recommend
\href{https://www.youtube.com/playlist?list=PL9D558D49CA734A02}{CS107}
(for an updated version see
\href{https://www.youtube.com/playlist?list=PLoCMsyE1cvdWivlV-39KKsBKUX-4DvraN}{this
playlist}) and
\href{https://www.youtube.com/playlist?list=PLu77E6J7s6Ko3Ft4XcOX1yKW6iX3eEFqS}{CS110}.
If you already feel comfortable with systems but want a recap, I would
recommend going through this short series by
\href{https://www.youtube.com/channel/UCseUQK4kC3x2x543nHtGpzw}{Brian
Will}: - \href{https://www.youtube.com/watch?v=9-KUm9YpPm0}{Hardware
Basics Video}, -
\href{https://www.youtube.com/watch?v=9GDX-IyZ_C8}{Operating Systems
Video}, - \href{https://youtu.be/xHu7qI1gDPA}{Unix System Calls 1},
\href{https://youtu.be/2DrjQBL5FMU}{Unix System Calls 2}, and -
\href{https://www.youtube.com/playlist?list=PLFAC320731F539902}{This
playlist on Unix terminals and shells}.
